\chapter{Conclusioni e Sviluppi Futuri} %\label{1cap:spinta_laterale}
% [titolo ridotto se non ci dovesse stare] {titolo completo}
%


In conclusione si può dire di aver effettuato una panoramica completa seppur basilare della struttura di un motore scacchistico.
Diversi algoritmi "storici" sono stati riproposti e ri-analizzati, mostrando la loro efficacia ed il motivo per cui sono diventati fondamentali ed è stata fornita la base ottimale per un motore di medio livello.\\
Infine, si elencano delle potenziali migliorie in grado di elevare un motore progettato seguendo questa
tesi da un motore amatoriale di medio livello ad un motore in grado di competere con i migliori al mondo.
\begin{itemize}
    \item \textbf{NNUE:} Una funzione di valutazione basata su NNUE è in grado, con un investimento temporale 
    molto breve, di battere anche le migliori funzioni di valutazione tradizionali. Per un motore 
    non particolarmente forte il passaggio a NNUE può comportare un guadagno di svariate centinaia di 
    punti elo.
    \item \textbf{Introduzione di ulteriori euristiche di ricerca:} Le euristiche di ricerca per un motore
    scacchistico sono numerose e nella tesi vengono trattate solo le principali, scavando più affondo
    nel mondo degli algoritmi di ricerca è possibile guadagnare diverse centinaia di punti elo.
    \item \textbf{Tuning del motore}: Un'altro fattore importante 
    nello sviluppo di un motore scacchistico consiste nel tuning dei suoi iperparametri, si tratta di un processo nel quale attraverso 
    molteplici iterazioni e campionamenti statistici si cerca il valore ottimale di ogni costante che detta il comportamento del motore.
    I guadagni attesi dal tuning di un motore dipendono dalla sub ottimalità dei parametri iniziali ma si aggirano solitamente sui 50-100 elo.
\end{itemize}
È possibile visualizzare un esempio pratico di alcune di queste migliorie e seguire lo sviluppo attivo di 
un motore scacchistico nella repository github del motore associato a questa tesi,
{\href{https://github.com/PGG106/Alexandria}{Alexandria}}.