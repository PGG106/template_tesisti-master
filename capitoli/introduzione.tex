\phantomsection
%\addcontentsline{toc}{chapter}{Introduzione}
\chapter{Introduzione}
\markboth{Introduzione}{}
% [titolo ridotto se non ci dovesse stare] {completo}

\section{Contesto applicativo} %\label{1sec:scopo}
Gli scacchi sono un gioco di strategia deterministico a somma zero e ad informazione completa  
che si svolge su una tavola quadrata formata da 64 caselle ,di due colori alternati,
detta scacchiera, sulla quale ogni giocatore contraddistinto da uno dei due colori (bianco o nero), dispone di 16 pezzi:un re, regina, due alfieri, due cavalli, due torri e otto pedoni.
obiettivo del gioco è dare scacco matto, ovvero minacciare la cattura del re avversario mentre esso non
ha modo di rimuovere il re dalla sua posizione di pericolo alla sua prossima mossa.



\section{Motivazioni e obiettivi}
Gli scacchi,gioco nato in india attorno al 600 d.C e campo di battaglia in uno dei più famosi scontri tra uomo e macchina 
(Kasparov vs Deep Blue 1996-1997), non hanno mai fallito nel saper cattivare  l'attenzione del grande pubblico nei loro 1400 anni di storia.
 \\È facile capire come sia possibile se ci si concentra su una delle caratteristiche fondamentali del 
 gioco degli scacchi  questa caratteristica è la \textbf{complessità}.\\In una partita di scacchi fin dalla prima mossa 
sono possibili 20 scelte, per la seconda  il totale di possibili combinazioni  sale a 400,
 dopo 5 mosse avremo 119,060,324 possibili risposte,in totale le possibili mosse di una partita si stimano attorno alle \(2^{155} \).
 \\Con uno spazio di ricerca cosi grande non dovrebbe stupire scoprire che è da quando esistono i computer che si cerca un modo
 di sfruttare la loro potenza di calcolo nel mondo degli scacchi.
 La nascita degli scacchi computazionali si deve al lavoro di Claude Shannon, famoso per i suoi innumerevoli contributi al 
 campo della teoria dell'informazione,egli, con il suo paper "Programming a Computer for Playing Chess"\cite{shannon} del 1950 ha gettato le
 basi per quello che oggi è il campo conosciuto come scacchi computazionali.
 \\Questa tesi nasce dalla volontà di esplorare questo vasto e interessante campo dell'informatica,e dal voler creare un testo
 in grado di guidare chiunque lo legga nella creazione di un motore scacchistico spiegando tutte le fasi della progettazione
 ed illustrando le possibili scelte che condizionano l'efficienza di un motore.

\section{Risultati ottenuti}
È stato ottenuto un motore scacchistico di medio livello che implementa tutte le componenti più importanti, in grado di essere 
capito anche da un novizio ma che allo stesso tempo funge da ottima base per lo sviluppo di un motore allo stato dell'arte,sono anche
stati raccolti dati sull'effetto di queste tecniche, utilizzati per produrre una prova della loro efficacia.



\section{Struttura della tesi}
La tesi è strutturata in 5 capitoli:
\begin{itemize}
\item Il primo funge da introduzione ed illustra a grandi linee il contenuto della tesi. 
\item Nel secondo viene affrontato il processo di progettazione di un motore scacchistico dal punto di vista teorico.
\item Nel terzo vengono illustrati gli effetti pratici delle scelte effettuate nel secondo capitolo
\item Il quarto è una panoramica sullo stato dell'arte dei motori scacchistici odierni.
\item Il quinto rappresenta uno specchio sui possibili sviluppi di un motore costruito a partire da questa tesi.
\end{itemize}
