\phantomsection
%\addcontentsline{toc}{chapter}{Introduzione}
\chapter{Introduzione}
\markboth{Introduzione}{}
% [titolo ridotto se non ci dovesse stare] {completo}

\section{Contesto applicativo} %\label{1sec:scopo}
Gli scacchi sono un gioco di strategia deterministico a somma zero e ad informazione completa  
che si svolge su una tavola quadrata formata da 64 caselle ,di due colori alternati,
detta scacchiera sulla quale ogni giocatore contraddistinto da uno di due colori
nero o bianco, dispone di 16 pezzi:un re, regina, due alfieri, due cavalli, due torri e otto pedoni.
obiettivo del gioco è dare scacco matto, ovvero minacciare la cattura del re avversario mentre esso non
ha modo di rimuovere il re dalla sua posizione di pericolo alla sua prossima semimossa.
Ulteriori informazioni verranno fornite nei successivi capitoli quando maggiori conoscenze di teoria 
si riveleranno necessarie per poter procedere allo sviluppo del motore.


\section{Motivazioni e obiettivi}
Gli scacchi,gioco nato in india attorno al 600 d.C,da gioco utilizzato nelle corti aristocratiche
 per rappresentare rapporti di potere a campo di battaglia tra uomo e macchina in uno dei primi e 
 più famosi tentativi di far superare ad una macchina l'intelletto umano (Kasparov vs Deep Blue 1996-1997)
,gli scacchi , non hanno mai  fallito nel saper cattivare  l'attenzione
 del grande pubblico nonostante abbiano ormai più di 1000 anni sulle spalle.
 \\Quello che agli occhi di un profano potrebbe sembrare un fenomeno  stranissimo
 è in realtà di facile spiegazione se ci si concentra su una delle caratteristiche fondamentali del 
 gioco degli scacchi  questa caratteristica è la \textbf{complessità},in una partita di scacchi fin dalla prima semimossa 
  sono possibili 20 scelte per la seconda semimossa il totale di possibili combinazioni  sale a 400,
 dopo 5 semimosse avremo 119,060,324 possibili risposte, le possibili mosse di una partita si stimano attorno alle \(2^{155} \).
 \\Con uno spazio di ricerca cosi grande non dovrebbe stupire sapere che è da quando esistono i computer che si cerca un modo
 di sfruttare la loro potenza di calcolo nel mondo degli scacchi.
 La nascita degli scacchi computazionali si deve al lavoro di Claude Shannon, famoso per i suoi innumerevoli contributi al 
 campo della teoria dell'informazione,egli, con il suo paper "Programming a Computer for Playing Chess" del 1950 ha gettato le
 basi per quello che oggi è il campo conosciuto come scacchi computazionali.
 \\Questa tesi nasce dalla volontà di esplorare questo vasto e interessante campo dell'informatica,e dal voler creare un testo
 in grado di guidare chiunque lo legga nella creazione di un motore scacchistico spiegando tutte le fasi della progettazione
 ed illustrando le possibili scelte che condizionano l'efficienza di un motore
 ,dato che la letteratura su questo fronte è  non particolarmente florida e soprattutto quasi esclusivamente in lingua inglese.




\section{Risultati ottenuti}

\section{Struttura della tesi}


\section{Prefazione} %\label{1sec:scopo}
Lo sviluppo di un motore scacchistico è fortemente influenzato dalle scelte progettuali,
una di queste è il linguaggio di programmazione che si vuole utilizzare,
le prestazioni di un motore possono essere fortemente influenzate dalla natura del linguaggio di
programmazione, in particolare l'utilizzo di un linguaggio interpretato e non compilato può impattare
notevolmente sulla velocità con la quale il nostro motore è in grado di elaborare le milioni di
posizioni con le quali dovrà avere a che fare in una singola partita.
Tutti gli esempi di codice all'interno di questa tesi saranno scritti nel linguaggio C, si consiglia
quindi di avere almeno una minima familiarità con tale linguaggio.Si segnalano comunque diversi
tool  per il linguaggio python per chi volesse approcciarsi a questo mondo utilizzando un
linguaggio più beginner friendly  quali:
\begin{itemize}
    \item \textbf{python-chess}: una libreria di python che contiene funzioni di libreria per la rappresentazione
          di scacchiera e pezzi e per la generazione e validazione delle mosse,utile se ci si vuole concentrare
          esclusivamente sulla parte di ricerca e di valutazione  di un motore scacchistico.
    \item \textbf{Sunfish}: un motore scacchistico per principianti scritto nel linguaggio python che
          in sole 111 linee di codice illustra ,in maniera semplificata, l'implementazione della
          gran parte dei concetti  chiave di un motore scacchistico.

\end{itemize}



