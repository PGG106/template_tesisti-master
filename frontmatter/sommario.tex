%\selectlanguage{italian}
\begin{abstract}

INSERIRE ABSTRACT
\\[1cm]
Gli scacchi sono da sempre un gioco tanto affascinante quanto complesso, ed è proprio questo li rende un terreno fertile per l'introduzione tecniche innovative di ricerca e valutazione. All'interno di questa tesi vengono affrontati gli algoritmi
più noti e vengono discusse le idee alla base della loro efficacia per infine accennare allo stato dell'arte del mondo dei motori scacchistici. In particolare vengono affrontati i concetti di mini-max e potatura alfa-beta, comuni alla quasi 
totalità degli algoritmi che tentano di risolvere giochi a turni quali scacchi e dama, e migliorie apportabili alla alfa-beta, applicate nel contesto degli scacchi computazionali ma generalizzabili 
per funzionare in un qualsiasi algoritmo mini-max (come NegaScout) o specifiche per il mondo degli scacchi computazionali e viene fornita un'idea generale su come costruire una funzione di valutazione adatta al gioco degli scacchi.
\end{abstract} 